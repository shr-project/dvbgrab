\chapter{Testování}

V rámci předmětu Styk člověka s počítačem jsem provedl i test tohoto systému.

Když byl systém provozován přibližně rok na Masarykově koleji ČVUT a měl registrováno cca 120 uživatelů, tak jsem rozeslal všem registrovaným neformální žádost. Žádost byla prvním krokem, chtěl jsem získat přehled, jaké se vyskytují chyby a nejasnosti v uživatelském rozhraní. 

Systém je koncipovaný převážně pro studentské prostředí, proto nebylo do testu zařazeno více osob ani osoby, které systém nikdy nepoužily (zdrojem e-mailových adres, na které byla rozesílána žádost, byla právě databáze uživatelů DVBgrabu.

Na e-mail odpovědělo sice pouze asi 10 osob, přesto byla nalezena některá vhodná vylepšení.

\section{Připomínky a reakce}

\textbf{Připomínka:} Všechny díly seriálu by mělo být možno objednat jedním tlačítkem.

\textbf{Reakce:} Nyní je možnost vyhledat v televizním programu podle názvu, takže všechny díly daného seriálu lze nalézt podle zadané části názvu a pak objednat snadno a rychle přímo ze~seznamu výsledků hledání. To je univerzálnější, a navíc není potřeba z programu detekovat, co je a~co není seriál a také není třeba zaznamenávat tyto seriálové požadavky mimo rozsah programu, který známe předem. Když objednám seriál, který má 4 díly v programu, který už je načten, a další 4 v programu, který se načte až při příští aktualizaci, tak bych musel udržovat další tabulku \quotedblbase dlouhodobých požadavků''.

\bigskip
 
\textbf{Připomínka:} Odkazy na nahrané pořady uveřejňovat po přihlášení přímo na stránkách. 

\textbf{Reakce:} Odkaz se původně pouze odeslal po nahrání na e-maily uživatelů, kteří si to objednali. Pokud e-mail smažou a odkaz si neuloží, tak nemají možnost se k záznamu dostat (kontrola IP adresy uživatele a generovaná část názvu pořadu). Tato připomínka byla nakonec implementována, protože značně zjednodušila použití systému. Seznam nahraných pořadů pro jednotlivé uživatele již byl na webu k dispozici, tak byl pouze doplněn o odkazy na jejich stažení. Také uživatelské adresáře jsou nyní zakládány s volbou na generování indexů (stačí zadat správný adresář, a pokud uživatel má právo přístupu podle IP adresy, tak je mu vypsán i jeho obsah).

\bigskip

\textbf{Připomínka:} Uživatelé si občas neuvědomují, že zadání správného e-mailu je podmínkou pro použití systému. Při registraci zadají nějakou hloupost a pak marně čekají, jak se dozví o~úspěšném nahrání pořadu.

\textbf{Reakce:} Tady by mohlo stačit výraznější varování v registračním formuláři, protože bohužel souhrnné informace o tom, jak to celé funguje (půl strany textu na úvodní stránce), asi nikdo nečetl.

\bigskip

\textbf{Připomínka:} Jeden člověk si stěžoval, že mu na webu vadí seznam nahrávek všech uživatelů a~přihlášeného uživatele. Prý by to mělo patřit do administrativní části webu.

\textbf{Reakce:} S tím nesouhlasím, protože pokud nějaký pořad nestihnu objednat, tak můžu zkontaktovat (e-mail, icq, jabber) osobu ze seznamu, která si jej objednala a zkusit se s ní domluvit na zpřístupnění.

\bigskip

\textbf{Připomínka:} Stejnému člověku se zdálo, že hlavní stránka a šest podstránek je zahlcení uživatele. 

\textbf{Reakce:} S tím také nesouhlasím, protože já jako uživatel a spousta dalších tyto podstránky používá, například seznam uživatelem nahraných pořadů získá s doplněním odkazů na jejich stažení další velký význam. Další výhodou zobrazovaného seznamu je lepší přehled uživatele, protože počet objednaných nahrávek za měsíc je omezen.

\bigskip

\textbf{Připomínka:} Lepší vysvětlení, co znamená \quotedblbase nagrabovat'' a \quotedblbase nagrabovat do TS''. 

\textbf{Reakce:} V úvodních informacích bylo napsáno, že po vybrání pořadu se objedná nahrávka s~komprimací do MPEG-4. Pokud uživatel chce provádět na nahrávce další úpravy, je pro něj výhodnější nahrávku nekomprimovat a zanechat ve formátu MPEG2-TS. Protože není na stránce se seznamem pořadů moc místa, tak je uveden pouze odkaz \quotedblbase nagrabovat'' a po potvrzení dialogu \quotedblbase Doopravdy chcete nahrát pořad abc?'' se odkaz změní na \quotedblbase zrušit grab'' a vedle se zobrazí \linebreak[4]\quotedblbase do TS'', čímž se zruší objednávka buď úplně nebo se zruší její komprimace (využitelné pro cca 5\% uživatelů). Navrhované řešení bylo nahradit jednoduchý potvrzovací dialog s \quotedblbase ANO'', \quotedblbase NE'' kompletní novou podstránkou s formulářem, který by zobrazoval detail zvoleného pořadu a obsahoval i popis, co je MPEG-4 a MPEG-TS a jak se o nahrání uživatel dozví. Na formuláři by pak mohla být tlačítka \quotedblbase Zruš grab'', \quotedblbase Grabuj do MPEG-4'', \quotedblbase Grabuj do MPEG-TS'' a \quotedblbase Storno''. Toto by sice bylo možné, ale na původním způsobu jsem oceňoval rychlost použití. Klik na název pořadu, enter na klávesnici pro výchozí volbu \quotedblbase ANO'' a už bylo objednáno. Takhle bude velmi zdržující čekat než se zobrazí formulář se spoustou opakujících se informací (popis formátů, způsob doručení odkazu pro stažení), zvolit správné tlačítko a počkat na opětovné vykreslení celého televizního programu. Celý systém se nakonec vyřešil volbou preferované komprese v~nastavení uživatelského profilu. Televizní program teď obsahuje pouze odkazy pro objednání a zrušení.

\bigskip

\textbf{Připomínka:} Jeden uživatel si také stěžoval na nemožnost změnit si dodatečně e-mail adresu.

\textbf{Reakce:} Tato možnost tam je \quotedblbase schována'' v menu nastavení. Bohužel mě nenapadá výstižnější název pro podstránku, na níž se nastavují parametry uživatelského účtu.

\bigskip

\section{Vyhodnocení implementovaných připomínek -- anketa}

Pro zjištění, jak uživatelé hodnotí provedené změny a některé pouze navrhované, jsem opět rozeslal na seznam registrovaných členů e-mail s žádostí o vyplnění krátké ankety, kterou jsem začlenil do webového rozhraní DVBgrabu. Každá odpověd se skládala z číselného ohodnocení 1=Rozhodně ANO až 5=Rozhodně NE a textového komentáře. V anketě svůj názor vyjádřilo přibližně 50 uživatelů.

\vfil
\pagebreak
Anketa dopadla takto:

\textbf{Otázka:} Odkazy na stažení grabů budou nyní dostupné přímo z menu \quotedblbase Moje graby''. Je tato vlastnost pro Vás přínosná?

\textbf{Odpovědi:} Všichni novou možnost uvítali, pouze 2 ohodnotili \quotedblbase spíše ano'', z toho jeden byl \quotedblbase věčným nespokojencem''.

\bigskip

\textbf{Otázka:} Nyní můžete objednávat seriály přes vyhledání v TV programu a v seznamu výsledků rychlým klikáním na jednotlivé díly. Chtěli byste mít i možnost objednat všechny díly seriálu jedním tlačítkem, které by ale dost problematicky poznávalo, co je a co není seriál, a také by muselo složitě řesit objednávání seriálů na dobu delší, než na jakou se načítá TV program. Potřebujete tuto funkčnost?

\textbf{Odpovědi:} Opět se většina shodla na ohodnocení \quotedblbase spíše ne'' a \quotedblbase nevím'', jen \quotedblbase věčný nespokojenec'' by požadoval i tuto funkci.

\bigskip
						
\textbf{Otázka:} Je ze současných textů už zřejmé, že správný e-mail je podmínkou použitelnosti \linebreak[4]DVBgrabu?

\textbf{Odpovědi:} Všichni potvrdili, že situace už je napravena. \quotedblbase Věčně nespokojený'' opět doplňuje, že s odkazy na graby z menu je už e-mail jenom doplňkovou podmínkou.
							
\bigskip

\textbf{Otázka:} Je ze současných textů už zřejmé, co myslím MPEG-4 a transport stream TS a co je výchozí volbou při grabování?

\textbf{Odpovědi:} Vyhodnocení je sporné, ale spíše pozitivní reakce, součástí komentářů byly i dva použitelné nápady na vylepšení.
							
\bigskip

\textbf{Otázka:} Příjdou Vám menu \quotedblbase seznam mých, plánovaných, hotových grabů'' zbytečná nebo je občas rádi využíváte? Takže tyto menu zachovat nebo ne?

\textbf{Odpovědi:} Všichni dotázaní požadují zachování, jen \quotedblbase věčně nespokojený'' přidává, že je nepotřebuje.
							
\bigskip

\textbf{Otázka:} Všimli jste si v menu volby \quotedblbase Nastavení'' a hodila se Vám?	

\textbf{Odpovědi:} Opět pozitivní odezva.
						
\bigskip

\textbf{Otázka:} Chybí Vám tu nějaká funkce? Případně do komentáře napište jaká.

\textbf{Odpovědi:} Nikdo nic nepostrádal.
				
\bigskip

\textbf{Otázka:} Libovolný vzkaz tvůrci DVBgrabu, náměty na vylepšení atd.

\textbf{Odpovědi:} Pár slov chvály a několik neuskutečnitelných návrhů.

\bigskip

\section{Vyhodnocení ankety}

Uživatelé byli dotazování na konkrétní body a konkrétní funkce, protože možnost vyjádřit se k~systému jako celku měli v prvním kole e-mailů. Také se projevil profil testované osoby \quotedblbase věčný nespokojenec'', který nebyl spokojen vcelku s ničím, ale vzhledem k tomu, že svým názorem značně vyčníval mimo průměr, mohl být jeho názor brán jako neobjektivní.

\bigskip

\section{Závěr z testování}

I přes účast pouze několika málo uživatelů se podařilo dát dohromady několik návrhů, které mohou daný systém zlepšit, zpřehlednit a urychlit. I tato jednoduchá forma testování se rozhodně vyplatila. Tato metoda vyžadovala minimální úsilí navíc. Také bylo objeveno několik drobných chyb v programu.

\bigskip

Kdyby byl test proveden na různorodém vzorku uživatelů (nejenom studenti technické školy), mohly se objevit i úplně jiné problémy, ale toto jsem netestoval, protože v době testu jsem nehodlal uveřejnit systém i pro jiné cílové skupiny uživatelů.
