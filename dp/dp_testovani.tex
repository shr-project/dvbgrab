\chapter{Testování}

V rámci předmětu Styk člověka s počítačem jsem provedl i test tohoto systému.

\vspace{10pt}

Když byl systém provozován přibližně rok na Masarykově koleji a měl registrováno cca 120 uživatelů, tak jsem rozeslal všem registrovaným neformální žádost. Žádost byla prvním krokem, chtěl jsem získat přehled, jaké se vyskytují chyby a nejasnosti v uživatelském rozhraní. 

\vspace{10pt}

Systém je zaměřený převážně na studentské prostředí, proto nebylo do testu zařazeno více osob ani osoby, které systém nikdy nepoužili (zdrojem e-mailových adres, na které byla rozesílána žádost, byla právě databáze uživatelů DVBgrabu.

\vspace{10pt}

Na email odpovědělo sice pouze asi 10 osob, přesto byly nalezeny některé vhodné vylepšení:

\vspace{10pt}

\section{Připomínky a reakce}

\begin{bf}Připomínka:\end{bf} Všechny díly seriálu by mělo být možno objednat jedním tlačítkem.

\begin{bf}Reakce:\end{bf} Nyní je možnost vyhledat v televizním programu podle názvu, takže všechny díly daného seriálu lze nalézt podle zadané části názvu, a pak objednat snadno a rychle přímo ze seznamu výsledků hledání. To je univerzálnější a navíc není potřeba z programu detekovat, co je a co není seriál a také není třeba zaznamenávat tyto seriálové požadavky mimo rozsah programu, který známe předem. Když objednám seriál, který má 4 díly v programu, který už je načten a další 4 v programu, který se načte až při příští aktualizaci, tak bych musel udržovat další tabulku "dlouhodobých požadavků".

\vspace{10pt}
 
\begin{bf}Připomínka:\end{bf} Odkazy na nahrané pořady uveřejňovat po přihlášení přímo na stránkách. 

\begin{bf}Reakce:\end{bf} Odkaz se původně pouze odeslal po nahrání na emaily uživatelů, kteří si to objednali. Pokud email smažou a odkaz si neuloží, tak nemají možnost se k záznamu dostat (kontrola IP adresy uživatele a generovaná část názvu pořadu). Tato připomínka byla nakonec implementována, protože značně zjednodušila použití systému a seznam nahraných pořadů pro jednotlivé uživatele již byl na webu k dispozici, tak byl pouze doplněn o odkazy na jejich stažení. Také uživatelské adresáře jsou nyní zakládány s volbou na generování indexů (stačí zadat správný adresář a pokud uživatel má právo přístupu podle IP adresy, tak je mu vypsán i jeho obsah).

\vspace{10pt}

\begin{bf}Připomínka:\end{bf} Uživatelé si občas neuvědomují, že zadání správného emailu je podmínkou pro použití systému. Při registraci zadají nějakou hloupost a pak marně čekají, jak se dozví o úspěšném nahrání pořadu.

\begin{bf}Reakce:\end{bf} Tady by mohlo stačit výraznější varování v registračním formuláři, protože bohužel souhrnné informace o tom, jak to celé funguje (půl strany textu na úvodní stránce), asi nikdo nečetl.

\vspace{10pt}

\begin{bf}Připomínka:\end{bf} Jeden člověk si stěžoval, že mu na webu vadí seznam nahrávek všech uživatelů a přihlášeného uživatele. Prý by to mělo patřit do administrativní části webu.

\begin{bf}Reakce:\end{bf} S tím nesouhlasím, protože pokud nějaký pořad nestihnu objednat, tak můžu například využít kontaktu (email, icq, jabber) na osobu ze seznamu, která si to objednala a zkusit se s ní domluvit na zpřístupnění.

\vspace{10pt}

\begin{bf}Připomínka:\end{bf} Stejnému člověku přišlo, že hlavní stránka + 6 podstránek je zahlcení uživatele. 

\begin{bf}Reakce:\end{bf} S tím také nesouhlasím, protože já jako uživatel a spousta dalších tyto podstránky používá a například seznam uživatelem nahraných pořadů získá s doplněním odkazů na jejich stažení další velký význam. Další výhodou zobrazovaného seznamu je lepší přehled uživatele, protože počet objednaných nahrávek za měsíc je omezen.

\vspace{10pt}

\begin{bf}Připomínka:\end{bf} Lepší vysvětlení, co znamená "nagrabovat" a "nagrabovat do TS". 

\begin{bf}Reakce:\end{bf} V úvodních informacích bylo napsáno, že po vybrání pořadu se objedná nahrávka s komprimací do MPEG-4. Pokud uživatel chce provádět na nahrávce další úpravy, je pro něj výhodnější nahrávku nekomprimovat a zanechat ve formátu MPEG2-TS. Protože není na seznamu pořadů moc místa, tak je uveden pouze odkaz "nagrabovat" a po potvrzení dialogu "Doopravdy chcete nahrát pořad abc?" se odkaz změní na "zrušit grab" a vedle se zobrazí "do TS", čímž se zruší objednávka buď úplně nebo se zruší její komprimace (využitelné pro cca 5\% uživatelů). Navrhované řešení bylo nahradit dialog s "ANO", "NE" kompletní novou podstránkou s formulářem, který by zobrazoval detail zvoleného pořadu a obsahoval i popis, co je MPEG-4, MPEG-TS a jak se o nahrání uživatel dozví. Na formuláři by pak mohli být tlačítka "Zruš grab", "Grabuj do MPEG-4", "Grabuj do MPEG-TS" a "Storno". Toto by sice bylo možné, ale na původním způsobu jsem oceňoval rychlost použití. Klik na název pořadu, enter na klávesnici pro výchozí volbu "ANO" a už bylo objednáno. Takhle než se zobrazí formulář se spoustou opakujících se informací (popis formátů, způsob doručení odkazu pro stažení), zvolit správné tlačítko a počkat na opětovné vykreslení celého televizního programu bude velmi zdržující. Celý systém se nakonec vyřešil volbou preferované komprese v nastavení uživatelského profilu. Televizní program teď obsahuje pouze odkazy pro objednání a zrušení.

\vspace{10pt}

\begin{bf}Připomínka:\end{bf} Jeden uživatel si také stěžoval na nemožnost změnit si dodatečně email adresu.

\begin{bf}Reakce:\end{bf} Tato možnost tam je "schována" v menu nastavení. Bohužel nenapadá mě výstižnější název pro podstránku, kde se nastavují parametry uživatelského účtu.

\vspace{10pt}

\section{Vyhodnocení implementovaných připomínek - anketa}

Pro zjištění, jak uživatelé hodnotí provedené změny a některé pouze navrhované, jsem opět rozeslal na seznam registrovaných členů email s žádostí o vyplnění krátké ankety, kterou jsem začlenil do webového rozhraní DVBgrabu. Každá odpověd se zkládala z číselného ohodnocení 1=Rozhodně ANO až 5=Rozhodně NE a textového komentáře.

\vspace{10pt}

Anketa dopada takto:

\vspace{10pt}

\begin{bf}Otázka:\end{bf}: Odkazy na stažení grabů budou nyní dostupné přímo z menu "Moje graby". Je tato vlastnost pro Vás přínosná?

\begin{bf}Odpovědi:\end{bf} Všichni novou možnost uvítali, pouze 2 ohodnotili "spíše ano", z toho jeden byl "věčným nespokojencem".

\vspace{10pt}

\begin{bf}Otázka:\end{bf} Nyní můžete objednávat seriály přes vyhledání v tv programu a v seznamu výsledků rychlým klikáním na jednotlivé díly. Chtěli byste mít i možnost objednat všechny díly seriálu jedním tlačítkem, které by ale dost problematicky poznávalo, co je a co není seriál a také by muselo složitě řesit objednávání seriálů na dobu delší než na kolik se načítá tv program. Potřebujete tuto funkčnost?

\begin{bf}Odpovědi:\end{bf} Opět se většina shodla na ohodnocení "spíše ne" a "nevím", jen "věčný nespokojenec" by požadoval i tuto funkci.

\vspace{10pt}
						
\begin{bf}Otázka:\end{bf} Je ze současných textů už zřejmé, že správný email je podmínkou použitelnosti DVBgrabu?

\begin{bf}Odpovědi:\end{bf} Všichni potvrdili, že situace už je napravena, "věčně nespokojený" opět doplňuje, že s odkazy na graby z menu je už email jenom doplňkovou podmínkou.
							
\vspace{10pt}

\begin{bf}Otázka:\end{bf} Je ze současných textů už zřejmé, co myslím MPEG-4 a transport stream TS a co je výchozí volbou při grabování?

\begin{bf}Odpovědi:\end{bf} Vyhodnocení je sporné, ale spíše pozitivní reakce, součástí komentářů byly i 2 použitelné nápady na vylepšení.
							
\vspace{10pt}

\begin{bf}Otázka:\end{bf} Příjdou Vám menu "seznam mých, plánovaných, hotových grabů" zbytečná nebo je občas rádi využíváte? Takže tyto menu zachovat nebo ne?

\begin{bf}Odpovědi:\end{bf} Všichni dotázaní požadují zachování, jen "věčně nespokojený" přidává, že je nepotřebuje.
							
\vspace{10pt}

\begin{bf}Otázka:\end{bf} Všimli jste si v menu volby "Nastavení" a hodila se Vám?	

\begin{bf}Odpovědi:\end{bf} Opět pozitivní odezva.
						
\vspace{10pt}

\begin{bf}Otázka:\end{bf} Chybí Vám tu nějaká funkce? Případně do komentáře napište jaká.

\begin{bf}Odpovědi:\end{bf} Nikdo nic nepostrádal.
				
\vspace{10pt}

\begin{bf}Otázka:\end{bf} Libovolný vzkaz tvůrci DVBgrabu, náměty na vylepšení atd.

\begin{bf}Odpovědi:\end{bf} Pár slov chvály a několik neuskutečnitelných návrhů.

\vspace{10pt}

\section{Vyhodnocení ankety}

Uživatelé byli dotazování na konkrétní body a konkrétní funkce, protože možnost vyjádřit se k systému jako celku měli v prvním kole emailů. Také se projevil profil testované osoby "věčný nespokojenec", který nebyl spokojen vcelku s ničím, ale vzhledem k tomu, že svým názorem značně vyčníval mimo průměr, mohl být jeho názor brán jako neobjektivní.

\vspace{10pt}

\section{Závěr z testování}

I přes účast pouze několika málo uživatelů se podařilo dát dohromady několik návrhů na vylepšení, které mohou daný systém zlepšit, zpřehlednit a urychlit. I tato jednoduchá forma testování se rozhodně vyplatila. Touto formou to vyžadovalo minimální úsilí navíc. Také bylo objeveno několik drobných chyb v programu.

\vspace{10pt}

Kdyby byl test proveden na různorodém vzorku uživatelů (nejenom studenti technické školy), mohly se objevit i úplně jiné problémy, ale toto jsem netestoval, protože zatím nehodlám uveřejnit systém i pro jiné cílové skupiny uživatelů.
