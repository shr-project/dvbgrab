\chapter{Formáty pro ukládání audio-video}

\vspace{10pt}

Audio-video data se ukládájí v souborovém a přenášejí v síťovém formátu, obecně tomu říkáme kontejner. Kontejner zajišťuje synchronizaci různých složek (audio, video, titulky) a může podporovat například různé kapitoly v rámci jednoho souboru (známé z DVD). Kontejner umí pracovat s určitými typy audio a video kodeků. Kodeky určují, jakým způsobem jsou data digitalně uložena. 

\vspace{10pt}

\section{Organizace ovlivňující audio-video formáty současnosti}

\vspace{10pt}

\textbf{MPEG}

Motion Pictures Expert Group je název standardizační skupiny ISO. V normách je v různých částech vždy obsažena definice jak kontejneru, tak audio i video kodeků.

\vspace{10pt}

\textbf{VCEG}

Video Coding Experts Group, skupina pro návrh audio-video standardu skupiny ITU-T.

\vspace{10pt}

\textbf{Firmy Microsoft, Apple a další}

\vspace{10pt}

\textbf{Neziskové organizace a dobrovolníci vytvářející svobodné implementace, obvykle kompatibilní s některým standardem.}

\vspace{10pt}

\section{Nejznámější kontejnery}

\vspace{10pt}

\textbf{MPEG-1}

Je nejstarším standardem, využívá se například u Video CD (VCD). Jeho kvalita je zhruba srovnatelná s kvalitou záznamu na analogové VHS kazetě. Součástí toho standardu je i známý audio kodek MP3, což je zkratka za MPEG-1 Part 3 Layer 3 (MPEG-1 Audio Layer 3). Přehrávání a nahrávání tohoto formátu je hardwarově nejméně náročné a také je to formát nejvíce kompatibilní.

\vspace{10pt}

\textbf{MPEG-2}

Nástupce MPEG-2 nově přináší podporu pro prokládané video a 2 různé kontejnery pro vkládání audio-video dat.

\vspace{10pt}

\textbf{MPEG-2 TS}

Transport stream neboli kontejner pro přenášení signálu po méně spolehlivém kanálu se používá například v DVB vysílání. 

\vspace{10pt}

\textbf{MPEG-2 PS}

Program stream je naopak navržen pro použití na spolehlivém médiu jako je DVD a SuperVideo CD (SVCD).

\vspace{10pt}

\textbf{MPEG-2 VOB}

Rozšíření MPEG-2 používané na DVD discích. Umožňuje definovat jednotlivé kapitoly, podpora pro titulky ve formátu VobSub a také pro ne-MPEG audio kodeky jako je často používaný AC3 pro prostorový zvuk 5.1.

\vspace{10pt}

\textbf{MPEG-3}

Původně definovaný pro televizní vysílání s vysokým rozlišením HDTV. Místo tohoto formátu se pro HDTV používá mírně vylepšený MPEG-2.

\vspace{10pt}

\textbf{MPEG-4 MP4}

Nejnovější formát z rodiny MPEG standardů. Největším rozdílem proti předchůdcům je použití kodeků s vysokou kompresí pro audio i video. Také je zahrnuta podpora pro systém ochrany autorských práv DRM (Digital Rights Management). Používá se nejčasteji pro uchování audio-video na počítači a tam, kde je třeba zajistit co nejmenší datový tok, třeba u DVB vysílání pro přenosná zařízení (DVB-H). Definuje 2 skupiny video kodeků s vysokou kompresí MPEG-4 ASP a MPEG-4 AVC.

\vspace{10pt}

\textbf{AVI}

Audio Video Interleave, souborový kontejner navržený firmou Microsoft podobný formátu MPEG-4.

\vspace{10pt}

\textbf{ASF} 

Advanced Systems Format, dříve Advanced Streaming Format, přenosový kontejner navržený firmou Microsoft.

\vspace{10pt}

\textbf{QuickTime}

Kontejner od firmy Apple. Používá přípony .mov a .qt. Byl základem pro tvorbu standardu MP4.

\vspace{10pt}

\textbf{OGG/OGM}

Svobodná varianta vyvinutá v organizaci Xiph.Org \cite{xiphURL}.

\vspace{10pt}

\textbf{Matroska}

Další svobodná varianta s popisem složek ve formátu EBML (Extensible Binary Meta Language - binární obdoba XML). Podobá se QuickTime, MP4 a ASF.

\vspace{10pt}

\textbf{3GP}

Převážně pro přenosná zařízení jako telefony a PDA navržený organizací 3GPP.

\vspace{10pt}

\textbf{NUT} (z projektu MPlayer/FFmpeg), \textbf{FLV} (video z Flash playeru od Adobe), \textbf{RealMedia}, ..

\vspace{10pt}

\section{Nejpoužívanější video kodeky}

\vspace{10pt}

\textbf{MPEG-1 Video}

Nejstarší a vhodný spíše pro video s nízkým rozlišením, podporuje pouze neprokládané video (progressive scan).

\vspace{10pt}

\textbf{MPEG-2 Video}

Kompatibilní s MPEG-1 Video, ale při toku více než 3Mbit/s již efektivnější a také podporuje prokládané video (interlaced) známé z televizního vysílání. Definuje různé profily (typy snímků P, I, B; rozložení YUV a kanálů) a úrovně (rozlišení, framerate, bitrate). Kombinací profilů a levelů získáme škálu různě kvalitních variant použitelných od bezdrátových zařízení po kvalitní zařízení pro vysoké rozlišení (HD).

\vspace{10pt}

\textbf{MPEG-4 ASP}

Advanced Simple Profile, odpovídá standardu VCEG H.263 (implementace XviD, DivX, 3ivx, QuickTime, libavcodec).

\vspace{10pt}

\textbf{MPEG-4 AVC}

Advanced Video Coding, vznikl spoluprací skupin MPEG a VCEG a odpovídá H.264 (implementace x264, libavcodec) používaný pro DVD s vysokým rozlišením (HD DVD) a pravděpodobně použitý pro BlueRay video disky, lze také použít pro vysílání DVB ve vysokém rozlišení.

\vspace{10pt}

\textbf{Theora}

Svobodný kodek kvalitativně srovnatelný s MPEG-4 ASP i MPEG-4 AVC.

\vspace{10pt}

\textbf{WMV}

Windows Media Video, odpověd Microsoftu na kodeky MPEG-4 ASP.

\vspace{10pt}

\textbf{RealVideo}, \textbf{MJPEG}, ..

\vspace{10pt}

\section{Nejpoužívanější audio kodeky}

\vspace{10pt}

\textbf{MPEG-1 Layer 2 Audio}

Nižší komprese.

\vspace{10pt}

\textbf{MPEG-1 Layer 3 Audio}

Vyšší ztrátová komprese, rozšířená podpora v přehrávačích, známý jako MP3.

\vspace{10pt}

\textbf{AAC}

Advanced Audio Coding, definovaný v MPEG-2 část 7 a MPEG-4 část 3 (podle implementace).

\vspace{10pt}

\textbf{AC3}

Adaptive Transform Coder 3, až 6 kanálový zvuk Dolby Digital.

\vspace{10pt}

\textbf{FLAC}

Free Lossless Audio Codec, svobodná alternativa bezeztrátového audio kodeku.

\vspace{10pt}

\textbf{Vorbis}

Svobodná náhrada kodeku MP3, také se ztrátovou kompresí.

\vspace{10pt}

\textbf{Speex}

Svobodná ztrátová komprese optimalizována pro uchování a přenos lidské řeči.

\vspace{10pt}

\textbf{WMA}, \textbf{RealAudio}, \textbf{ATRAC}, ..

