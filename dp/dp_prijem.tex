\chapter{Příjem multicastového vysílání z vnější sítě}

Na Internetu je dostupných i mnoho dalších televizních kanálů, které jsou vysílány pro veřejnost. Příjem je ale problematický, protože musí být zaručeno směrování multicastu z veřejného Internetu až do naší sítě. Je třeba vyřešit kompatibilitu multicastových démonů na routerech založených na Linuxu s Cisco routery. Tak aby mezi v naší síti a v síti poskytovatele připojení byla na každém sousedním routeru spuštěna služba pro směrování multicast dat. Případně lze vytvořit tunel, kterým jsou některé multicastové skupiny vysílány jako klasické unicast pakety. Směrování multicast paketů a správu skupin obvykle poskytuje pim démon na cisto routerech, na linuxových pak například pimd nebo mrouted.

\vspace{10pt}

Pro operační systém Linux se nejčastěji používají 2 multicastové směrovací démoni pimd a mrouted. Bohužel se už nevyvíjí a jejich současné verze rozhodně nejsou dokonalé.

\vspace{10pt}

\textbf{PIMd}

\vspace{5pt}

Poslední verze která se používá je alpha verze z roku 1999. Podporuje směrovací protokol DVMRP a MOSPF. Hodí se pro více využité multicastové skupiny nebo pro sítě s velkým přenosovým pásmem.

\vspace{10pt}

Pokud jsou skupiny využívané zřídka tak toto schéma přestává být efektivní. Proto vznikla odnož pim démona:

\vspace{10pt}

\textbf{PIM-SM}

\vspace{5pt}

Pim démon v sparse módu. Udržuje tabulku odběratelů a zdrojů určité skupiny a podle toho vytváří distribuční stromy. Kořen distribučních stromů se nazývá "Rendezvous Point".

\vspace{10pt}

Implementace pim démona s otevřeným kódem

Zastaralý: \textbf{Pimd USC site} (samostatný PIM-SM + úprava jádra systému)

Zastaralý: \textbf{PIM-SM GateD} implmentace od ISI.

\textbf{PIM-DM GateD} implementace z Oregonské univerzity

\textbf{Pimd-dense} samostatná implementace z Oregonské univerzity

\textbf{PIM-SM} implementace z XORP projektu (implementace software směrovačů s otevřeným kódem)

\vspace{10pt}

Více viz \cite{pimdURL}.

\vspace{10pt}

\textbf{MROUTEd}

\vspace{5pt}

Poslední používaná verze je beta z roku 1999. Mrouted implementuje také DVMRP směrovací protokol. Podporuje také tunely skrz směrovače, které nepodporují multicast. Více viz \cite{mroutedURL}.
