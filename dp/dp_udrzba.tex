
\section{Údržba záznamového serveru}

O údržbu se stará automaticky spouštěný skript clean.php.

\vspace{10pt}

\textbf{Mazání neaktivních uživatelů}

Maže uživatele, kteří se po dobu několika dní (volba user\_inactivity\_limit) nepřihlásili k DVBgrabu a ani si nic neobjednávali. To zajistí zrušení uživatele a uvolnění jeho IP adresy po jeho odstěhovaní z koleje (pokud předem sám nepoužije funkci na okamžité zrušení účtu). Také to ruší všechny jeho požadavky a jeho uživatelský adresář. O zrušení účtu se uživateli pošle email.

\vspace{10pt}

\textbf{Aktualizace uživatelských účtů}

Změna hesla nebo stahovací IP adresy uživatelem nastavuje v databázi datum poslední změny. Tento skript při každém spuštění pro všechny uživatele, kteří měnili nastavení mezi posledním spuštěním skriptu a aktuálním časem, přegeneruje .htaccess soubory v jejich adresářích. Poté posune datum poslední aktualizace na aktuální čas.

\vspace{10pt}

\textbf{Kontrola neznámých uživatelských účtů}

Skript zkontroluje, zda pro všechny adresáře existuje odpovídající databázový uživatel, pokud ne, mohou být smazány, případně alespoň vypsány na obrazovku.

\vspace{10pt}

\textbf{Mazání nepotřebných .ts souborů}

Všechny graby se ukládají do souboru s .ts koncovkou. Při každém spuštění skriptu je u těchto souborů zkontrolováno, zda k nim existuje nejaký nevyřízený požadavek na komprimaci. Pokud ne, je .ts soubor smazán. Pokud ano, je do logovacího souboru zapsáno, který formát ještě chybí připravit.

\vspace{10pt}

\textbf{Kontrola volného místa}

V grabovacím adresáři se sleduje množství využitého prostoru a volné místo. 

Nejstarší nahrávky se začnou mazat při překročení maximální povolené velikosti grabovacího adresáře (podle grab\_storage\_size), nebo pokud je volného místa méně než je opět nakonfigurovaná konstanta (grab\_storage\_min\_size). 
Pokud je nutné mazat i graby, které jsou nagrabovány kratší dobu než je plánované uchovávání grabů (grab\_history), je správci zaslán varovný email.

\vspace{10pt}

\textbf{Mazání starých dat}

V databázi nepotřebujeme uchovávat televizní program několik let dozadu a zárověn je dobré zobrazovat na webu i hotové graby po jejich smazání. Proto jsou z databáze smazány jen data starší než je trojnásobek doby, po kterou se uchovávají graby (3 x grab\_history dní). Mažou se jak záznamy o uživatelských požadavcích, tak požadavky na formát, tak vlastní graby a televizní program.

\section{Údržba uživatelů}

Občas je nutné ručně rušit uživatele, kteří se například odstěhují a nezruší přitom svůj účet. Pokud jejich IP adresu dostane nový uživatel, který by se chtel zaregistrovat na DVBgrabu dřív než uplyne lhůta na automatické zrušení původního účtu z důvodu neaktivity, registrace selže z důvodu konfliktu IP. V těchto případech je nejlepším řešením ručně nastavit datum poslední aktivity původního účtu například rok dozadu a tím je zajištěno zrušení účtu údržbovým skriptem, včetně zrušení jeho adresáře a odeslání informačního emailu.

\vspace{10pt}

Pro hromadné informování uživatelů je možné použít skript backend/send\_user\_info.php, ve kterém upravíme text zprávy a jeho spuštěním rozešleme zprávu včetně přihlašovacích údajů každému uživateli DVBgrabu. Toto je vhodné použít například po konverzi databáze z dvbgrabu verze 1 na verzi 2, kdy došlo ke změnám v pravidlech pro znaky v uživatelských jménech.

\vspace{10pt}

\section{Údržba backend skriptů}

\vspace{10pt}

Veškeré spouštění a zastavování grabovacích nebo komprimačních smyček by se mělo provádět, přes skript dvbgrab\_service, který je součástí instalačního balíků. V tomto skriptu zkontrolujeme cestu k adresáři s backend skripty.

\vspace{10pt}

Skript dvbgrab\_service podporuje tyto parametry:

\vspace{10pt}

\textbf{start} - nastartuje záznamovou i komprimační smyčku

\textbf{startg} - nastartuje pouze záznamovou smyčku

\textbf{starte} - nastartuje pouze komprimační smyčku

\textbf{stop} - zastaví obě smyčky

\textbf{stopg} - zastaví pouze záznamovou smyčku

\textbf{stope} - zastaví pouze komprimační smyčku

\textbf{restart} - restartuje obě smyčky

\textbf{restartg} - restartuje pouze záznamovou smyčku

\textbf{restarte} - restartuje pouze komprimační smyčku

\textbf{status} - zobrazí, jestli smyčky běží a poté pro jednotlivé komprimační formáty seznam grabů, které jsou ve frontě a také, které graby se právě ukládají

\vspace{10pt}

Zastavování komprimační složky je doplněno o obnovování fronty na komprimaci (skriptem backend/encode\_queue\_restore.php), což po zastavení smyčky vrátí stav požadavků ze stavu "Komprimuje se" na stav "Nahraný", tudíž po opětovném spuštění smyčky se mohou znovu vybrat ke zpracování.

\vspace{10pt}

Zobrazit frontu na komprimaci a aktuálně ukládané graby lze i ručně pomocí skriptu backend/encode\_queue\_print.php.

\vspace{10pt}

Pro spouštění a vypínání DVBgrabu při startu a zastavování operačního spouštění máme skript, který tento dvbgrab\_service volá, měl bý být v /etc/init.d/dvbgrab.
