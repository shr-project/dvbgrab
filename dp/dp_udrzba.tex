\section{Údržba záznamového serveru}
O údržbu se stará automaticky spouštěný skript clean.php.

\subsection{Mazání neaktivních uživatelů}
Maže uživatele, kteří se po dobu několika dní (podle volby user\_inactivity\_limit) nepřihlásili \linebreak[4]k~DVBgrabu a ani si nic neobjednali. To zajistí zrušení uživatele a uvolnění jeho IP adresy po~jeho odstěhovaní z koleje (pokud předem sám nepoužije funkci na okamžité zrušení účtu). Také to ruší všechny jeho požadavky a jeho uživatelský adresář. O zrušení účtu se uživateli pošle e-mail.

\subsection{Aktualizace uživatelských účtů}
Při změně hesla nebo stahovací IP adresy uživatelem je v databázi nastaveno nové datum poslední změny. Tento skript při každém spuštění přegeneruje .htaccess soubory pro všechny uživatele, kteří měnili svá nastavení mezi posledním spuštěním skriptu a aktuálním časem. Poté posune datum poslední aktualizace na aktuální čas.

\subsection{Kontrola neznámých uživatelských účtů}
Skript zkontroluje, zda pro všechny adresáře existuje odpovídající databázový uživatel, pokud ne, mohou být smazány, případně alespoň vypsány na obrazovku.
\vfil
\pagebreak
\subsection{Mazání nepotřebných .ts souborů}
Všechny graby se ukládají do souboru s .ts koncovkou. Při každém spuštění skriptu je u těchto souborů zkontrolováno, zda k nim existuje nejaký nevyřízený požadavek na komprimaci. Pokud ne, je .ts soubor smazán. Pokud ano, je do logovacího souboru zapsáno, který formát ještě chybí připravit.

\subsection{Kontrola volného místa}
V grabovacím adresáři se sleduje množství využitého prostoru a volné místo. 

Nejstarší nahrávky se začnou mazat při překročení maximální povolené velikosti grabovacího adresáře (podle grab\_storage\_size), nebo pokud je volného místa méně, než je opět nakonfigurovaná konstanta (grab\_storage\_min\_size). 

Pokud je nutné mazat i graby, které jsou nagrabovány kratší dobu, než je plánované uchovávání grabů (grab\_history), je správci zaslán varovný e-mail.

\subsection{Mazání starých dat}
V databázi nepotřebujeme uchovávat televizní program několik let dozadu a zárověn je dobré zobrazovat na webu i hotové graby po jejich smazání. Proto jsou z databáze smazány jen data starší než je trojnásobek doby, po kterou se uchovávají graby (3 x grab\_history dní). Mažou se jak záznamy o uživatelských požadavcích, tak požadavky na formát, tak vlastní graby a televizní program.

\section{Ruční údržba uživatelů}
Občas je nutné ručně rušit uživatele, kteří se například odstěhují a nezruší přitom svůj účet. Pokud jejich IP adresu dostane nový uživatel, který by se chtěl zaregistrovat na DVBgrabu dřív, než uplyne lhůta na automatické zrušení původního účtu z důvodu neaktivity, registrace selže z důvodu konfliktu IP. V těchto případech je nejlepším řešením ručně nastavit datum poslední aktivity původního účtu například rok dozadu, a tím je zajištěno zrušení účtu údržbovým skriptem, včetně zrušení jeho adresáře a odeslání informačního e-mailu.

Pro hromadné informování uživatelů je možné použít skript backend/send\_user\_info.php, ve~kterém upravíme text zprávy a jeho spuštěním rozešleme zprávu včetně přihlašovacích údajů každému uživateli DVBgrabu. Toto je vhodné použít například po konverzi databáze z DVBgrabu verze 1 na verzi 2, kdy došlo ke změnám v pravidlech pro znaky v uživatelských jménech.

\section{Údržba backend skriptů}
Veškeré spouštění a zastavování grabovacích nebo komprimačních smyček by se mělo provádět přes skript dvbgrab\_service, který je součástí instalačního balíku. V tomto skriptu zkontrolujeme cestu k adresáři s backend skripty.

Skript dvbgrab\_service podporuje tyto parametry:
\bitem
\item \textbf{start} -- nastartuje záznamovou i komprimační smyčku
\item \textbf{startg} -- nastartuje pouze záznamovou smyčku
\item\textbf{starte} -- nastartuje pouze komprimační smyčku
\item\textbf{stop} -- zastaví obě smyčky
\item\textbf{stopg} -- zastaví pouze záznamovou smyčku
\item\textbf{stope} -- zastaví pouze komprimační smyčku
\item\textbf{restart} -- restartuje obě smyčky
\item\textbf{restartg} -- restartuje pouze záznamovou smyčku
\item\textbf{restarte} -- restartuje pouze komprimační smyčku
\item\textbf{status} -- zobrazí, jestli smyčky běží. Poté pro jednotlivé komprimační formáty seznam grabů, které jsou ve frontě a také, které graby se právě ukládají
\eitem

Zastavování komprimační složky je doplněno o obnovování fronty na komprimaci (skriptem backend/encode\_queue\_restore.php), což po zastavení smyčky vrátí stav požadavků ze stavu \quotedblbase Komprimuje se'' na stav \quotedblbase Nahraný'', díky tomu se mohou po opětovném spuštění smyčky opět vybrat ke zpracování.

Zobrazit frontu na komprimaci a aktuálně ukládané graby lze i ručně pomocí skriptu \linebreak[4] backend/encode\_queue\_print.php.

Pro spouštění a vypínání DVBgrabu při startu a zastavování operačního spouštění máme skript, který tento dvbgrab\_service volá. Měl bý být v /etc/init.d/dvbgrab.
