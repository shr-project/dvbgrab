\chapter{Instalace a údržba}

\section{Potřebné knihovny a pomocné programy}
\section{Stažení DVBgrabu}
\section{Založení databáze}
\section{Konfigurace DVBgrabu}

Budeme potřebovat stáhnout zdrojové kódy, nejaktuálnější verze je v subversion repozitáři, který lze stáhnout příkazem
\begin{small}\begin{verbatim}svn export --username anonymous http://martinja.mk.cvut.cz/svn/dvbgrab/trunk\end{verbatim}\end{small}
pokud máme nainstalovaný balík subversion.
K dispozici jsou i samostatné instalační balíky různých verzí např. dvbgrab-20051128.tgz.

\vspace{10pt}

Instalační balík obsahuje adresář, který je třeba nahrát na server, kde bude provozováno webové rozhraní do adresáře apache (dokument root), obvykle jako /var/www/dvbgrab a podadresář backend patří nahrát na záznamový server, tam je opět asi nejlepší založit nového uživatele např. dvbgrab a do jeho domovského adresáře nakopírovat soubory z adresáře backend.

\vspace{10pt}

Konfigurace se provádí spuštěním configure.sh pro povolení zápisu do konfiguračních souborů a poté ve webovém prohlížeči na stránce http://název\_serveru/dvbgrab/setup.php , když je konfigurace dokončena, je třeba spustit skript secure.sh, který nastaví zpět práva konfiguračního souboru config.php jen pro čtení a tento konfigurační soubor je třeba zkopírovat i na záznamový server aby měl stejnou verzi.

\vspace{10pt}

Na webovém serveru je ještě třeba zajistit instalaci ADOdb a to buď jako podadresář /var/www/dvbgrab nebo pokud jsme instalovali z distribučního balíku tak bude někde v /usr. Cestu k ADOdb je třeba ještě donastavit v dblib.php a to jak pro webový, tak pro záznamový server.

\vspace{10pt}

Také se může hodit změnit jazykovou mutaci, celého webového rozhraní. Připraveny jsou soubory pro českou a anglickou lokalizaci. Přepnutí se provede v souboru language.inc.php, plánováno je přepínání na úrovni uživatele (např. volbou ikony vlajky a zapamatováním v cookie prohlížece), případně detekování preferovaného jazyka z nastavení locale prohlížeče.

\vspace{10pt}

Na databázovém serveru je třeba založit databáze s tabulkami. Zakládací skript je připraven pro MySQL a PostgreSQL v sql/mysql.sql resp. sql/postgres.sql. Taky založíme nového databázového uživatele a přiřadíme mu práva na tyto tabulky. U MySQL je důležité zachovat kódování textů ISO-8859-2, které webové rozhraní předpokládá.

\vspace{10pt}

Na některém serveru je také třeba do cronu přidat automatické spouštění aktualizace televizního programu (obvykle se o to stará server s databází). Skript zaznam.php se spouští pravidelně každý týden (třeba v sobotu) a parametrem je počet zpracovávaných dnů (obvykle je k dispozici na 10 dní) 

\vspace{10pt}

Na záznamovém serveru v adresáři s obsahem backend je třeba zajistit spouštění 2 skriptů, grab\_loop.php a encode\_loop.ph. To lze zajistit buď definováním nové služby a přiřazením do spouštění ve výchozím runlevelu. Nebo některé verze cronu to umí přes příznak restart. Také je dobré v cronu zajistit denní spouštění send\_daily\_report.php, které posílá denně seznam nahraných pořadů.

\vspace{10pt}

Také je potřeba nastavit apache, aby v nějakém adresáři v document root mohl zakládat adresáře jednotlivým uživatelům a v těch adresářích vždy založí i .htaccess soubor, který omezuje přístup k této složce jen na IP adresu, ze které se uživatel registroval. Do těchto uživatelských adresářů se potom umisťují symbolické odkazy, které mají částečně generované názvy a odkazují do adresáře se všemi hotovými pořady (obvykle třeba /pub/grab).

\vspace{10pt}

Poslední úprava je potřeba ve skriptu dvbgrab, kde je nutno nastavit k odpovídajícím názvům kanálů odpovídající IP adresy (multicastové skupiny). Toto se ale možná přesune do definice pořadu v databázi.

\vspace{10pt}
\section{The maintainance of recording server}
The maintainance is performed by automatically started script clean.php.

\subsection{Removing inactive users}
Removes users, who didn't sign up to DVBgrab in period of several days (wrt. config option user\_inactivity\_limit) and they didn't request anything. This will provide removing user and releasing his IP address after his moving from the dorm (unless he uses himself the function for immediate removing his account in advance). It removes also all his requests and his user's address book. An e-mail with information about the removing account will be delivered to the user.

\subsection{Users' accounts updating}
When the user changes the password or distributing IP address, new date of the last change is set in database. This script will always regenerate .htaccess files with every start for all users, who changed their settings between the last running of the script and actual time. Then it will move date of the last updating to the actual time.

\subsection{Control of unknown users' accounts}
Script will control, if there is appropriate database user for all directories, if not, they can be removed, eventually at least printed on the screen.

\subsection{Removing needless .ts files}
All records are saved to the file with .ts extension. With every start of the script these files are controlled, if there exists any unsatisfied request for encoding. If not, .ts file is removed. If yes, it is written to the log file, which format is missing yet to prepare.

\subsection{Control of free space}
In records directory an amount of used space and free space is watched. 

The oldest records start to be removed, if maximum enabled size of records directory is exceeded (wrt. grab\_storage\_size) or if there is less free space than configurated constant (grab\_storage\_min\_size).     

If it is necessary to remove also the records, which are finished for shorter time than planned records keeping (grab\_history), a warning e-mail is sent to admin.

\subsection{Removing old data}
In a database we don't need to keep television program, that is several years old and it is good to display on the web also the finished records after removing. That's why only data older than three times the period for keeping records (3 x grab\_history days) are removed from database. Users' requests are removed as well as format requests, own records and television program.

\section{The manual maintainance of users}
Sometimes it is necessary to remove manually those users, who e.g. move from the dorm and don't remove their accounts. If their IP address is given to a new user, who would like to register in DVBgrab before the period for automatic removing original account is exceeded by reason of inactivity, then registration fails by reason of IP conflict. In these cases the best solution is to set manually a date of the last activity of original account e.g. a year ago, and so removing account is provided by the maintainance script, including removing his address book and sending informational e-mail.

For spamming of registered users script backend/send\_user\_info.php might be used, in which we modify the text of the message and by its start we send the message including sign up data to every user of DVBgrab. This is suitable to use e.g. after the conversion of the database from DVBgrab version 1 to version 2, when the changes in the rules for characters in users names were performed.

\section{The maintainance of backend scripts}
Every start and stop of recording or encoding loops should be provided by script dvbgrab\_service, which is a component of installation package. In this script we check the path to backend directory.

Script dvbgrab\_service supports these parameters:
\bitem
\item \textbf{start} -- starts recording and encoding loop
\item \textbf{startg} -- starts only recording loop
\item\textbf{starte} -- starts only encoding loop
\item\textbf{stop} -- stops both loops
\item\textbf{stopg} -- stops only recording loop
\item\textbf{stope} -- stops only encoding loop
\item\textbf{restart} -- restarts both loops
\item\textbf{restartg} -- restarts only recording loop
\item\textbf{restarte} -- restarts only encoding loop
\item\textbf{status} -- shows, if the loops are still running. Then the list of records (which are in the queue and also those which are just saved) is shown for every encoding format.
\eitem

Stopping of encoding loop is supplemented with restoring of the queue for encoding (by the script backend/encode\_queue\_restore.php). It means that after stop of the loop request status returns from status \quotedblbase Encoding'' to status \quotedblbase Ready'', so they can be queued again after restart of the loop.

Displaying the queue for encoding and currently saved records can be provided also manually by means of the script backend/encode\_queue\_print.php.

We have startup script for operating system (/etc/init.d/dvbgrab), which uses dvbgrab\_service when starting and stopping DVBgrab as system service.

